% Define glossary entries with \newglossaryentry{<label>}{<settings>}
\newglossaryentry{latex}{
        name=latex,
        description={Is a mark up language specially suited for 
scientific documents}
}

\newglossaryentry{pileup}{
        name=pileup,
        description={}
}

\newglossaryentry{luminosity}{
        name=luminosity,
        description={}
}

\newglossaryentry{jet}{
        name=jet,
        description={}
}

\newglossaryentry{svj}{
        name=semi-visible jet,
        description={}
}

\newglossaryentry{particleflow}{
        name=Particle Flow algorithm,
        description={}
}

\newglossaryentry{antikt}{
        name=anti-\kt algorithm,
        description={}
}

% Define acronyms with \newacronym{<label>}{<abbrv>}{<full>}
\newacronym{cern}{CERN}{Organisation Europ\'{e}enne pour la Recherche Nucl\'{e}aire/European Organization for Nuclear Research}

\newacronym{lhc}{LHC}{Large Hadron Collider}

\newacronym{cms}{CMS}{Compact Muon Solenoid}

\newacronym{atlas}{ATLAS}{A Toroidal LHC ApparatuS}

\newacronym{ecal}{ECAL}{Electromagnetic Calorimeter}

\newacronym{hcal}{HCAL}{Hadron Calorimeter}

\newacronym{l1}{L1}{Level-1}

\newacronym{l1t}{L1T}{Level-1 Trigger}

\newacronym{hlt}{HLT}{High-Level Trigger}

\newacronym{pu}{PU}{pileup}

\newacronym{pf}{PF}{Particle Flow}

\newacronym{ps}{PS}{Proton Synchrotron}

\newacronym{sps}{SPS}{Super Proton Synchrotron}

\newacronym{mev}{MeV}{megaelectron volt}

\newacronym{gev}{GeV}{gigaelectron volt}

\newacronym{tev}{TeV}{teraelectron volt}

\newacronym{sm}{SM}{Standard Model}

\newacronym{qcd}{QCD}{Quantum Chromodynamics}

\newacronym{mc}{MC}{Monte Carlo}

\newacronym{vbf}{VBF}{vector boson fusion}

% To reference glossary terms and acronyms in the document, see https://en.wikibooks.org/wiki/LaTeX/Glossary#Using_defined_terms for commands
