%
% File: introduction.tex
% Author: Eshwen Bhal
% Description: Introductory chapter.
%
\let\textcircled=\pgftextcircled
\chapter{Introduction}
\label{chap:intro}

% Hard to describe dark matter as an object. Do I call it a 'substance', 'thing', 'mass'? Can refer to it as 'non-luminous' if struggling for other adjectives

\initial{T}he universe, in all its vastness, structure, natural laws and chaos, is comprised of only three principal components: visible matter, the ingredients of stars, planets and life, is the only one we interact with on a regular basis; dark energy, a force or manifestation of something even more mysterious, responsible for the accelerating expansion of the universe, is almost entirely unknown; and dark matter, a substance invisible in all sense of the word, that binds galaxy together and influences large scale structure in the cosmos, is the focus of this thesis.


\section{Evidence for dark matter}
\label{sec:intro_dm_evidence}

% Taken directly from second year report. Tidy up and improve
The observable universe contains a major component that is not explained by conventional \acrfull{sm} physics. Making up approximately 25.8\% of the energy density of the universe \cite{2016AnA...594A..13P}, the astrophysical evidence suggests that there exists a massive, neutral constituent that has a significant gravitational influence on the visible matter we observe. Labelled ``dark matter", this entity has captured the interest of many scientists. Those in particle physics have developed theories and experimental searches in an attempt to understand the characteristics of dark matter.

Dark matter was thought to be created in the hot, early universe when the thermal background allowed its spontaneous pair production. When the universe expanded and cooled, a thermal freeze out occurred: the average temperature became too low to allow significant production \cite{Baldes:2017gzw}. Matter became further separated and the dark matter annihilation rate decreased, leaving a ``thermal relic". These remaining particles were attracted via gravity, the only known force by which dark matter interacts. They formed filaments throughout the universe, and the potential wells they induced allowed the progenitors of galaxies to form within.

Although dark matter cannot be directly studied with conventional astronomy, there are several independent astrophysical observations that suggest its existence. The rotation curves of most galaxies are roughly flat \cite{1996MNRAS-281-27P}, contrary to the expected Keplerian curve ($v \propto 1/\sqrt{r}$) from solely baryonic matter. On the galactic scale, dark matter is sprinkled in a roughly spherical halo that spans beyond the observable disc. The inclusive dark matter mass increases linearly \cite{2009arXiv0901.0632E} to compensate for the decline expressed by visible matter \cite{1970ApJ-160-811F,1992AandA-256-19B}. Weak gravitational lensing of galaxies can cause images to appear distorted from dark matter between the galaxy and observer warping its local spacetime \cite{2010GReGr..42.2177H}.

From these observations, several properties of dark matter can be inferred. It is electrically neutral as it does not interact with light. It is ``cold" (non-relativistic), implying its rest mass energy is much greater than the thermal background in the universe. Current estimates suggest its mass is at the \acrfull{gev} or \acrfull{tev} scale \cite{Lowette:2014yta,1742-6596-651-1-012023,Hong:2017avi}. If it were on the neutrino scale -- and therefore relativistic -- it would be too diffuse to condense and allow galaxy formation. This supports the idea of ``bottom-up'' structure formation in the universe; smaller galaxies form around dark matter clumps, then merge to form larger structures \cite{doi:10.1093-mnras-183.3.341}. This also asserts that dark matter is stable, at least on the timescale of the age of the universe.


\subsection{Overview of dark matter searches}
\label{subsec:intro_dm_searches}

Whilst all current evidence has been astrophysical, determining the properties of dark matter falls into the realm of particle physics. Of the three types of dark matter searches, its production from high-energy collisions is being probed at the \acrshort{lhc}. Protons are collided at energies sufficient to produce the heavy particles that existed in the high-temperature early universe. The \acrfull{cms} experiment utilises its general purpose detector to allow physicists to search for dark matter in different theoretical frameworks.

Despite the \acrlong{sm} providing precise predictions of three of the four fundamental forces and the particles that they interact with, it does not substantiate the existence of dark matter. Several theories that are beyond (\acrshort{bsm}), or extend, the \acrlong{sm} can accommodate dark matter candidates such as sterile neutrinos \cite{doi:10.1142/S0218301313300191}, axions \cite{1981PhLB..104..199D}, and Kaluza-Klein states \cite{Han:1998sg}.

I have so far searched for dark matter in the context of \acrfull{susy} \cite{Martin:1997ns}. The theory introduces a spin symmetry that predicts a fermionic superpartner for each boson, and vice versa. If the \acrfull{lsp} is stable and electrically neutral, it would provide a promising dark matter candidate. Expected \acrshort{susy} particle decays produce the \acrshort{lsp} and, typically, hadronic jets because of the initial state particles. As \glspl{lsp} are undetectable, a reconstructed event from a detector would show a momentum imbalance. It contains ``missing" transverse energy (\glssymbol{met}) which is required to satisfy energy and momentum conservation. So the characteristics of \acrshort{susy} in a collider would be high \etmiss from the \glspl{lsp}, and several jets. But as no hint of supersymmetry has been found, other theories and simplified models from more complete theories have been considered. These are discussed later on.

There is significant motivation to study dark matter from a wider, as well as a more personal, viewpoint. It is important to understand how the universe operates, and dark matter opens up the potential for new physics that improves our understanding of nature. My personal interests include the blend of particle physics and astrophysics, and the opportunity to discover and add to humanity's collective wisdom. With a projected 130 \fbinv from the \acrshort{lhc} Run-2 at a centre-of-mass energy \comruntwo, there is great potential to constrain some of the properties of dark matter.

\newpage

\begin{easylist}[itemize]
\ListProperties(Style*=-- , FinalMark={)}, Margin=0.5cm)
& Discuss dark matter: motivation, evidence for its existence (and why it can't be neutrinos/dead stars/interstellar debris, etc.), detection methods and how we can probe it at the LHC (production). Should most of this stuff go in the introduction instead?
& Briefly outline particle accelerators and their function, the fact that we can use them to potentially discover dark matter or infer more of its properties, and the models that will be discussed in more detail to try and achieve this outcome.
& The introduction probably doesn't need to be too long, maybe only a few pages. Compare length with other people's theses (ask Ben Krikler for a copy of his, look at Alex's and Lana's).
\end{easylist}

\newpage

Here is some text, just to check how it's displayed. blah blah blah blah blah blah blah blah blah blah blah blah blah blah blah blah blah blah blah blah blah blah blah blah blah blah blah blah blah blah blah blah blah blah blah blah blah blah blah blah blah blah blah blah blah blah blah blah blah blah blah blah blah blah blah blah blah blah blah blah blah blah blah blah blah blah blah blah blah blah blah blah blah blah blah blah blah blah blah blah 

\newpage

Doing the same to check both sides of the paper (for when it's bound).

Also testing glossaries: \gls{pileup}, \acrlong{lhc}, \acrshort{lhc}, \acrfull{lhc}.

Also testing references: \cite{CMS-PAPER-SUS-15-005-published} (article), \cite{tagkey1984quarksandleptons} (book), \cite{Lisanti:2016jxe} (inproceedings), \cite{CMS-PAS-HIG-18-008} (techreport).

Testing numbers: 1234567890 (normal), $1234567890$ (math), \si{1234567890} (from siunitx).

Testing alphabet: The quick brown fox jumps over the lazy dog

Testing math characters compared to normal text: \Pqb-tag, $b$-tag, \emph{b}-tag, $\Pqb b\mathrm{b}\text{b}$\emph{b}b, $\Pqc c\mathrm{c}\text{c}$\emph{c}c.

Testing equations: $\sfrac{1}{2} \rho \Delta \phi \mathcal{L}$ (inline)
\begin{equation}
B(P) = \frac{\mu_0}{4\pi} \int \frac{I \times \hat{r}}{\bar{r}^2}\mathrm{d}r \ \text{(equation environment)}
\end{equation}

Testing symbols/macros: \eV, \MeV, \GeV, \TeV, \pt, \ptmiss, \met, \HT, \mht, \mt, \aDark, \rinv, \mqdark, \doubleMuCr, \doubleLepMass, \alphat, \ttbarpjets, \wtolnupjets, \LSP.

%=========================================================
