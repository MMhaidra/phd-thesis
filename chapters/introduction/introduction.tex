%
% File: introduction.tex
% Author: Eshwen Bhal
% Description: Introductory chapter.
%
\let\textcircled=\pgftextcircled
\chapter{Introduction}
\label{chap:intro}

% Hard to describe dark matter as an object. Do I call it a 'substance', 'thing', 'mass'? Can refer to it as 'non-luminous' if struggling for other adjectives

\initial{T}he universe, in all its vastness, structure, natural laws and chaos, is comprised of only three principal components: visible matter, the ingredients of stars, planets and life, is the only one we interact with on a regular basis; dark energy, a force or manifestation of something even more mysterious, responsible for the accelerating expansion of the universe, is almost entirely unknown; and dark matter, a substance invisible in all sense of the word, that binds galaxy together and influences large scale structure in the cosmos, is the focus of this thesis.


\section{Evidence for dark matter}
\label{sec:intro_dm_evidence}

% Taken directly from second year report. Tidy up and improve. Probably only need to give an overview of things in the introduction, and save the nitty-gritty for the theory section
The observable universe contains a major component that is not explained by conventional \acrfull{sm} physics. Making up approximately 25.8\% of the energy density of the universe \cite{2016AnA...594A..13P}, the astrophysical evidence suggests that there exists a massive, neutral constituent that has a significant gravitational influence on the visible matter we observe. Labelled ``dark matter", this entity has captured the interest of many scientists. Those in particle physics have developed theories and experimental searches in an attempt to understand the characteristics of dark matter.

Dark matter was thought to be created in the hot, early universe when the thermal background allowed its spontaneous pair production. When the universe expanded and cooled, a thermal freeze out occurred: the average temperature became too low to allow significant production \cite{Baldes:2017gzw}. Matter became further separated and the dark matter annihilation rate decreased, leaving a ``thermal relic". These remaining particles were attracted via gravity, the only known force by which dark matter interacts. They formed filaments throughout the universe, and the potential wells they induced allowed the progenitors of galaxies to form within.

Although dark matter cannot be directly studied with conventional astronomy, there are several independent astrophysical observations that suggest its existence. The rotation curves of most galaxies are roughly flat \cite{1996MNRAS-281-27P}, contrary to the expected Keplerian curve ($v \propto 1/\sqrt{r}$) from solely baryonic matter. On the galactic scale, dark matter is sprinkled in a roughly spherical halo that spans beyond the observable disc. The inclusive dark matter mass increases linearly \cite{2009arXiv0901.0632E} to compensate for the decline expressed by visible matter \cite{1970ApJ-160-811F,1992AandA-256-19B}. Weak gravitational lensing of galaxies can cause images to appear distorted from dark matter between the galaxy and observer warping its local spacetime \cite{2010GReGr..42.2177H}.

From these observations, several properties of dark matter can be inferred. It is electrically neutral as it does not interact with light. It is ``cold" (non-relativistic), implying its rest mass energy is much greater than the thermal background in the universe. Current estimates suggest its mass is at the \acrfull{gev} or \acrfull{tev} scale \cite{Lowette:2014yta,1742-6596-651-1-012023,Hong:2017avi}. If it were on the neutrino scale -- and therefore relativistic -- it would be too diffuse to condense and allow galaxy formation. This supports the idea of ``bottom-up'' structure formation in the universe; smaller galaxies form around dark matter clumps, then merge to form larger structures \cite{doi:10.1093-mnras-183.3.341}. This also asserts that dark matter is stable, at least on the timescale of the age of the universe.


\subsection{Overview of dark matter searches}
\label{subsec:intro_dm_searches}

Whilst all current evidence has been astrophysical, determining the properties of dark matter falls into the realm of particle physics. Of the three types of dark matter searches, its production from high-energy collisions is being probed at the \acrshort{lhc}. Protons are collided at energies sufficient to produce the heavy particles that existed in the high-temperature early universe. The \acrfull{cms} experiment utilises its general purpose detector to allow physicists to search for dark matter in different theoretical frameworks.

Despite the \acrlong{sm} providing precise predictions of three of the four fundamental forces and the particles that they interact with, it does not substantiate the existence of dark matter. Several theories that are beyond (\acrshort{bsm}), or extend, the \acrlong{sm} can accommodate dark matter candidates such as sterile neutrinos \cite{doi:10.1142/S0218301313300191}, axions \cite{1981PhLB..104..199D}, and Kaluza-Klein states \cite{Han:1998sg}.

I have so far searched for dark matter in the context of \acrfull{susy} \cite{Martin:1997ns}. The theory introduces a spin symmetry that predicts a fermionic superpartner for each boson, and vice versa. If the \acrfull{lsp} is stable and electrically neutral, it would provide a promising dark matter candidate. Expected \acrshort{susy} particle decays produce the \acrshort{lsp} and, typically, hadronic jets because of the initial state particles. As \glspl{lsp} are undetectable, a reconstructed event from a detector would show a momentum imbalance. It contains ``missing" transverse energy (\glssymbol{met}) which is required to satisfy energy and momentum conservation. So the characteristics of \acrshort{susy} in a collider would be high \etmiss from the \glspl{lsp}, and several jets. But as no hint of supersymmetry has been found, other theories and simplified models from more complete theories have been considered. These are discussed later on.

There is significant motivation to study dark matter from a wider, as well as a more personal, viewpoint. It is important to understand how the universe operates, and dark matter opens up the potential for new physics that improves our understanding of nature. My personal interests include the blend of particle physics and astrophysics, and the opportunity to discover and add to humanity's collective wisdom. With a projected 130 \fbinv from the \acrshort{lhc} Run-2 at a centre-of-mass energy \comruntwo, there is great potential to constrain some of the properties of dark matter.


% Taken directly from my lab book. Tidy up and improve. Merge with previous section
\begin{easylist}[itemize]
\ListProperties(Style*=-- , FinalMark={)})

& Thought to be made of a neutral, weakly interacting particle. Because it's neutral (and thought to be elementary), it cannot couple to the electromagnetic field, so there's no way to detect it through conventional astronomy.

& Galaxies form and condense in dark matter wells because the baryonic matter does not produce a gravitational field strong enough to keep ahold of ordinary matter.

& Dark matter is not super dense (like a black hole around the centre of a galaxy). It is more "sprinkled" in a roughly-spherical halo throughout a galaxy. \cite{1970ApJ-160-811F,1992AandA-256-19B} Its average density is still much greater than that of the normal matter in the galaxy because it's the dark matter that leads to the flat rotation curve of most galaxies. \cite{1996MNRAS-281-27P} If there were only baryonic matter, the rotation curve would drop off as $\propto 1/\sqrt{r}$ as Kepler's laws state. Whilst the surface density of dark matter does decrease with radius -- so the density in a thin spherical shell around the galaxy decreases -- its inclusive mass increases linearly (see Figure 1 in \cite{2009arXiv0901.0632E}) to compensate for the Keplerian drop off in the baryonic matter (think Gauss' law). \cite{1972ApJ-176-1G,2006AJ-132-2685M} The dark matter "orbits" the centre of the galaxy \underline{with} the stars, but in some models has a lower velocity because it's angular momentum doesn't dissipate via collisions and collapse into a disc like baryonic matter does (because collisions would result in annihilation). In other models, it's corotational with the stars.

& Dark matter particles must be stable because they've existed for billions of years and have allowed galaxies to form in their potential wells.

& It is thought that dark matter was produced thermally in the early universe (when it was hot and therefore easier to create heavy particles). Dark matter particles could (and did) annihilate, but as the universe expanded and cooled the dark matter became more diffuse. They didn't annihilate as often and the dark matter we see today is a "thermal relic" and is what's left over. Some models suggest the parameter $x = m_{\mathrm{DM}}/T \sim 20$ at freeze out. \cite{Lisanti:2016jxe} 

& The current leading dark matter candidate is a WIMP (Weakly Interacting Massive Particle), possibly a neutral supersymmetric particle (neutralino) like a higgsino, photino, zino, etc. One of the motivations for WIMPs are that, using the current values of the dark matter density in the universe and approximations for the annihilation cross section, dark matter could self-interact at the electroweak scale to produce Standard Model particles. \cite{Kamionkowski:1997zb} As the EW scale can be readily accessed at colliders like the LHC, we could detect DM signatures via pair production then annihilation or indirect searches via solely annihilation.

& WIMP annihilation could produce two showers of quarks, which would normally be observed as pions and high energy photons (like gamma rays). The photons may be of a continuum -- from hadronisation and radiation of the decay products of annihilation -- or contain features (internal radiation from the propagator in the interaction or from loop-level processes).

& Because WIMPs are stable (at least on the timescale of the current age of the Universe), they wouldn't decay into other particles when produced from accelerator collisions. So you would detect them (indirectly) by looking for MET and by looking for visible particles recoiling against the WIMPs.

& The mediator (force-carrying particle, like the gauge bosons) for dark matter -- between dark matter particles or the dark matter-Standard Model particle interactions -- may be a scalar (spin-0, like the Higgs boson) or pseudoscalar (reverses parity under a Lorentz transformation, like the pion) boson. [Support] for a pseudoscalar over a scalar mediator comes from the Feynman diagrams for DM annihilation into, e.g., $b$-quarks. With a scalar mediator, the vertex factors and the propagator term lead to cancellations in the cross section equation in the low-velocity limit.

& The mediator for dark matter may be heavier than the dark matter particle itself (like with the $W^{\pm}$ and $Z$ bosons being heavier than most of the particles they mediate), maybe 2x heavier or more so than the DM particle. The mediator could decay via DM pair production, so it makes sense that it would be at least twice as heavy.

& There's no consensus on whether dark matter is fermionic or bosonic. If fermionic, it may be either a Dirac fermion (particle is distinct from its antiparticle, like the electron and positron) or Majorana fermion (particle is the same as its antiparticle, like the neutrino is \underline{suspected} to be). If it were Majorana, dark matter could annihilate with itself, making discoveries via indirect searches more likely.

& At the LHC, monojets are used most prominently to look for dark matter particles. But multijet plus \etmiss might provide better sensitivity and constraints (particularly if the mediator is pseudoscalar).

& The Coma Cluster of galaxies seems to contain a very high concentration of dark matter (mass-to-light ratio of 400 solar masses per solar luminosity). See \cite{Yozin:2015mla}.

& \underline{Many} dark matter candidates include a few supersymmetric particles (the neutralino being the most widely studied), sterile neutrinos \cite{doi:10.1142/S0218301313300191}, axions, Kaluza-Klein states \cite{Han:1998sg} (which are excitations of Standard Model fields in extra dimensions), etc.

& Dark matter has to be cold (non-relativistic), as opposed to hot (relativistic), implying dark matter particles are reasonably heavy. If dark matter was light, it would have a lot of energy when produced in the early universe and would be relativistic (so hot). But if it were hot, it would be too diffusive to allow galaxies to form. However, because of the seemingly bottom-up nature of structure formation in the Universe \cite{doi:10.1093-mnras-183.3.341} (smaller galaxies form first around clumps of dark matter, then orbit and merge with other galaxies to form clusters and larger elliptical galaxies), dark matter must be cold so it doesn't diffuse too much and can clump to allow galaxy formation.

& The different aspects of dark matter searches: indirect (annihilation), direct (scattering from SM particles), and collider (production). Try to find/make a good image that showcases this (Feynman diagram with arrows for each type of detection).

& Some good results showcasing dark matter masses and that of its mediator from different analyses and decay channels are at \cite{CMS-DP-2016-057}. Particularly figures 4 and 6, which I used in my poster for the PGR conference.

& The 2015 results from Planck estimate the dark matter content of the universe to be 25.8\%. It displays it in terms of $\Omega_c h^2 = 0.1186$, where $h = 0.678$ (the Hubble constant, in units of km s$^{-1}$ Mpc$^{-1} / 100$), giving $\Omega_c = 0.258$ as the cold dark matter density \cite{2016AnA...594A..13P}.

& Dark matter cannot be solely neutrinos because the flux densities of neutrinos (from stars, as well as the cosmic neutrino background \cite{weinberg2008cosmology}) are precise and well-known, and due to the upper limit on neutrino masses \cite{1742-6596-718-2-022013}, are too small to account for the dark matter content in the universe. Because the neutrinos have such a small mass, they would be highly relativistic in the early universe (as they are today, despite it being cooler now, making them slower) and so could only contribute to hot dark matter \cite{Quigg:2008ab}. But as experiments show, the vast majority of dark matter must be cold.

& MOND (Modified Newtonian Dynamics) is one theory that tries to explain dark matter, and can be constrained to explain galactic rotation curves and other astrophysical phenomena attributed to it. However, any certain strand that tries to explain one observation usually falls flat when trying to explain others. It also doesn't work at all the scales to which we can observe the effects of General Relativity, almost confirming that GR is the correct description of gravity and MOND is a failure.

& LUX (Large Underground Xenon experiment) and LZ (LUX-Zeplin) are direct detection experiments that search explicitly for WIMP dark matter. LUX uses an underground liquid xenon tank to detect WIMPs interacting with ordinary matter, the scattering producing photons and electrons of specific energies. LZ is a collaboration between the LUX and ZEPLIN groups, and will have a highly-sensitive WIMP detector over a large range of masses, once completed.

& Evidence for non-luminous, \emph{non-baryonic} matter (an argument for those who ask why dark matter can't be neutrons, etc.):

One can use Doppler shifts of light emitted from galaxies in clusters, and therefore determine their masses. Then using the mass-to-light ratios of these galaxies and clusters (e.g., Bullet Cluster \cite{BulletClusterDMevidence}), one can determine that most of the mass comes from non-luminous matter \cite{cox2016universal}.

One can also use the Cosmic Microwave Background to calculate the average photon and neutrino (mass/energy) densities and Big Bang Nucleosynthesis calculations to determine the baryonic matter density. These can be compared to other measurements (e.g, mass-to-light ratios averaged across the universe) and reveal the discrepancy \cite{cox2016universal}.

Neutrons can't contribute to dark matter because isolated neutrons are unstable, decaying in a matter of minutes \cite{PDGbooklet2010}. They decay into protons and electrons. Being charged, they interact strongly with light and therefore contribute to the luminous matter in the universe.

\end{easylist}


% Taken directly from my lab book. Tidy up and improve. Merge with previous section. Look through these papers to see which ones give compelling interpretations and motivation for dark matter
Papers to look at regarding SUSY and dark matter:

\begin{easylist}[itemize]
\ListProperties(Style*=, FinalMark={)})
& \cite{dmsearcheslhc2015}
& \cite{dmbenchmarkearlylhcrun2}
& \cite{CMS-PAS-EXO-12-055}
& \cite{Martin:1997ns}
& \cite{CMS-PAS-SUS-15-005}
& \cite{Aitchison:2005cf}
& \cite{Ellis:2002mx}
& \cite{Murayama:2007ek}
& \cite{Peskin:2007nk}
& \cite{Goodman:2010ku}
& \cite{PhysRevLett.115.181802}
& \cite{CMS:2016pod}
& \cite{Bertone:2004pz}
\end{easylist}


\newpage

\begin{easylist}[itemize]
\ListProperties(Style*=-- , FinalMark={)}, Margin=0.5cm)
& Discuss dark matter: motivation, evidence for its existence (and why it can't be neutrinos/dead stars/interstellar debris, etc.), detection methods and how we can probe it at the LHC (production). Should most of this stuff go in the introduction instead?
& Briefly outline particle accelerators and their function, the fact that we can use them to potentially discover dark matter or infer more of its properties, and the models that will be discussed in more detail to try and achieve this outcome.
& The introduction probably doesn't need to be too long, maybe only a few pages. Compare length with other people's theses (ask Ben Krikler for a copy of his, look at Alex's and Lana's).
\end{easylist}

\newpage

Here is some text, just to check how it's displayed. blah blah blah blah blah blah blah blah blah blah blah blah blah blah blah blah blah blah blah blah blah blah blah blah blah blah blah blah blah blah blah blah blah blah blah blah blah blah blah blah blah blah blah blah blah blah blah blah blah blah blah blah blah blah blah blah blah blah blah blah blah blah blah blah blah blah blah blah blah blah blah blah blah blah blah blah blah blah blah blah 

\newpage

Doing the same to check both sides of the paper (for when it's bound).

Also testing glossaries: \gls{pileup}, \acrlong{lhc}, \acrshort{lhc}, \acrfull{lhc}.

Also testing references: \cite{CMS-PAPER-SUS-15-005-published} (article), \cite{tagkey1984quarksandleptons} (book), \cite{Lisanti:2016jxe} (inproceedings), \cite{CMS-PAS-HIG-18-008} (techreport).

Testing numbers: 1234567890 (normal), $1234567890$ (math), \si{1234567890} (from siunitx).

Testing alphabet: The quick brown fox jumps over the lazy dog

Testing math characters compared to normal text: \Pqb-tag, $b$-tag, \emph{b}-tag, $\Pqb b\mathrm{b}\text{b}$\emph{b}b, $\Pqc c\mathrm{c}\text{c}$\emph{c}c.

Testing equations: $\sfrac{1}{2} \rho \Delta \phi \mathcal{L}$ (inline)
\begin{equation}
B(P) = \frac{\mu_0}{4\pi} \int \frac{I \times \hat{r}}{\bar{r}^2}\mathrm{d}r \ \text{(equation environment)}
\end{equation}

Testing symbols/macros: \eV, \MeV, \GeV, \TeV, \pt, \ptmiss, \met, \HT, \mht, \mt, \aDark, \rinv, \mqdark, \doubleMuCr, \doubleLepMass, \alphat, \ttbarpjets, \wtolnupjets, \LSP.

%=========================================================
