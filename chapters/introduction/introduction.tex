%
% File: introduction.tex
% Author: Eshwen Bhal
% Description: Introductory chapter.
%
\let\textcircled=\pgftextcircled
\chapter{Introduction}
\label{chap:intro}

\initial{T}he universe, in all its vastness, structure, natural laws and chaos, is comprised of only three principal components: visible matter, the ingredients of stars, planets and life, is the only one we interact with on a regular basis; dark energy, a force or manifestation of something even more mysterious, responsible for the accelerating expansion of the universe, is almost entirely unknown; and dark matter, a substance invisible in all sense of the word, that binds galaxy together and influences large scale structure in the cosmos, is the topic of this thesis.

\begin{easylist}[itemize]
\ListProperties(Style*=-- , FinalMark={)}, Margin=0.5cm)
& Describe dark matter. If the main talking points regarding motivation, evidence for its existence, etc. will be discussed in theory chapter, then only briefly describe here.
& Briefly outline particle accelerators and their function, the fact that we can use them to potentially discover dark matter or infer more of its properties, and the models that will be discussed in more detail to try and achieve this outcome.
& The introduction probably doesn't need to be too long, maybe only a few pages. Compare length with other people's theses (ask Ben Krikler for a copy of his, look at Alex's and Lana's).
\end{easylist}

%=======
\section{Section}
\label{sec:sec01}

Begins a section.

\subsection{Subsection}
\label{subsec:subsec01}

Begins a subsection.


\newpage

Here is some text, just to check how it's displayed. blah blah blah blah blah blah blah blah blah blah blah blah blah blah blah blah blah blah blah blah blah blah blah blah blah blah blah blah blah blah blah blah blah blah blah blah blah blah blah blah blah blah blah blah blah blah blah blah blah blah blah blah blah blah blah blah blah blah blah blah blah blah blah blah blah blah blah blah blah blah blah blah blah blah blah blah blah blah blah blah 

\newpage

Doing the same to check both sides of the paper (for when it's bound).

Also testing glossaries: \gls{pileup}, \acrlong{lhc}, \acrshort{lhc}, \acrfull{lhc}.

Also testing references: \cite{CMS-PAPER-SUS-15-005-published} (article), \cite{tagkey1984quarksandleptons} (book), \cite{Lisanti:2016jxe} (inproceedings), \cite{CMS-PAS-SUS-15-005} (techreport).

Testing numbers: 1234567890 (normal), $1234567890$ (math), \si{1234567890} (from siunitx).

Testing alphabet: The quick brown fox jumps over the lazy dog

Testing math characters compared to normal text: \Pqb-tag, $b$-tag, \emph{b}-tag, $\Pqb b\mathrm{b}\text{b}$\emph{b}b, $\Pqc c\mathrm{c}\text{c}$\emph{c}c.

Testing equations: $\sfrac{1}{2} \rho \Delta \phi \mathcal{L}$ (inline)
\begin{equation}
B(P) = \frac{\mu_0}{4\pi} \int \frac{I \times \hat{r}}{\bar{r}^2}\mathrm{d}r \ \text{(equation environment)}
\end{equation}

Testing symbols/macros: \eV, \MeV, \GeV, \TeV, \pt, \ptmiss, \met, \HT, \mht, \mt, \aDark, \rinv, \mqdark, \doubleMuCr, \doubleLepMass, \alphat, \ttbarpjets, \wtolnupjets, \LSP.

%=========================================================