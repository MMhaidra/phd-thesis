\newcommand{\keywords}[1]{\par\noindent{\small{\bf Keywords:} #1}} % Defines keywords small section
\newcommand{\pgftextcircled}[1]{                                                                    % Defines encircled text
    \setbox0=\hbox{#1}%
    \dimen0\wd0%
    \divide\dimen0 by 2%
    \begin{tikzpicture}[baseline=(a.base)]%
        \useasboundingbox (-\the\dimen0,0pt) rectangle (\the\dimen0,1pt);
        \node[circle,draw,outer sep=0pt,inner sep=0.1ex] (a) {#1};
    \end{tikzpicture}
}

% My caption style
\newcommand{\mycaption}[2][\@empty]{
	\captionnamefont{\scshape} 
	\changecaptionwidth
	\captionwidth{0.9\linewidth}
	\captiondelim{.\:} 
	\indentcaption{0.75cm}
	\captionstyle[\centering]{}
	\setlength{\belowcaptionskip}{10pt}
	\ifx \@empty#1 \caption{#2}\else \caption[#1]{#2}
}

% My subcaption style
\newcommand{\mysubcaption}[2][\@empty]{
	\subcaptionsize{\small}
	\hangsubcaption
	\subcaptionlabelfont{\rmfamily}
	\sidecapstyle{\raggedright}
	\setlength{\belowcaptionskip}{10pt}
	\ifx \@empty#1 \subcaption{#2}\else \subcaption[#1]{#2}
}

% An initial of the very first character of the content
\usepackage{lettrine}
\newcommand{\initial}[1]{%
	\lettrine[lines=3,lhang=0.33,nindent=0em]{
		\color{gray}
     		{\textsc{#1}}}{}}

% Generic
\renewcommand{\_}{\texttt{\char`_}}
\newcommand{\plusjets}{\ensuremath{+ \text{jets}}\xspace}
\newcommand{\jone}{\ensuremath{\mathrm{j}_1}\xspace}
\newcommand{\jtwo}{\ensuremath{\mathrm{j}_2}\xspace}
\newcommand{\PVec}{\HepParticle{V}{}{}\xspace} % couldn't find command for generic V boson so created one

% Rationale for when to use \text{} and when to use \mathrm{} in physics symbols:
%     In math mode, wrapping something \text{} will render it in the main document font (i.e., garamond from the 'garamondx' package).
%     In math mode, wrapping something in \mathrm{} will render it upright in the math font (i.e., the font from the 'amsfonts' package).
%     \text{} should be used for things like control regions where the text from the macro is supposed to flow into the main text. Then it looks natural and consistent
%     \mathrm{} should be used for subscripts and superscripts in symbols so it looks natural and consistent in that context. Even for sub/superscripts longer than one letter, e.g., the "min" in biased delta phi, it looks more natural if the font matches the rest of the symbol and equation.

% Particles are typeset using the 'hepnames' package. If given the option 'italic', it will render the particles in italics. So if I want to change style, only need to edit the line that imports the package, not the macros.

% Higgs to inv. modes
\newcommand{\ttH}{\ensuremath{\Pqt\Pqt\PH}\xspace}
\newcommand{\ggF}{\ensuremath{\cPg\cPg\mathrm{F}}\xspace}
\newcommand{\ggH}{\ensuremath{\cPg\cPg\PH}\xspace}
\newcommand{\VH}{\ensuremath{\PVec\PH}\xspace}
\newcommand{\WH}{\ensuremath{\PW\PH}\xspace}
\newcommand{\WplusH}{\ensuremath{\PWp\PH}\xspace}
\newcommand{\WminusH}{\ensuremath{\PWm\PH}\xspace}
\newcommand{\ZH}{\ensuremath{\PZ\PH}\xspace}

% Higgs to inv. control regions
\newcommand{\singleMuCr}{\ensuremath{\Pgm \plusjets}\xspace}
\newcommand{\doubleMuCr}{\ensuremath{\Pgm\Pgm \plusjets}\xspace}
\newcommand{\singleEleCr}{\ensuremath{\Pe \plusjets}\xspace}
\newcommand{\doubleEleCr}{\ensuremath{\Pe\Pe \plusjets}\xspace}
\newcommand{\singlePhotonCr}{\ensuremath{\Pphoton \plusjets}\xspace}

% Variables/physics symbols
\newcommand{\doubleMuMass}{\ensuremath{m_{\Pgm\Pgm}}\xspace}
\newcommand{\doubleEleMass}{\ensuremath{m_{\Pe\Pe}}\xspace}
\newcommand{\doubleLepMass}{\ensuremath{m_{\ell\ell}}\xspace}

\newcommand{\etmiss}{\MET}
\newcommand{\met}{\MET}
\newcommand{\htmiss}{\mht}
\newcommand{\alphat}{\ensuremath{\alpha_{\mathrm{T}}}\xspace}
\newcommand{\alphaT}{\alphat}
\newcommand{\biasedDPhi}{\ensuremath{\Delta\phi^*_{\mathrm{min}}}\xspace}
\newcommand{\pT}{\pt}
\renewcommand{\mT}{\ensuremath{M_{\mathrm{T}}}\xspace}  % changing transverse mass from m_T to M_T
\newcommand{\mt}{\mT}

\newcommand{\mTsup}[1]{\ensuremath{\mT^{#1}}\xspace}  % to add a superscript in mT. Call like \mTsup{foo} to get m_T^{foo}
\newcommand{\mtMuon}{\ensuremath{\mTsup{\Pgm}}\xspace}
\newcommand{\mtElectron}{\ensuremath{\mTsup{\Pe}}\xspace}
\newcommand{\mjj}{\ensuremath{m_{\mathrm{jj}}}\xspace}
\newcommand{\ptsup}[1]{\ensuremath{\pt^{#1}}\xspace}  % to add a superscript in pT
\newcommand{\etasub}[1]{\ensuremath{\eta_{#1}}\xspace} % to add a subscript in eta
\newcommand{\abseta}{\ensuremath{\lvert \eta \rvert}\xspace}
\newcommand{\absetasub}[1]{\ensuremath{\lvert \eta_{#1} \rvert}\xspace}
\newcommand{\nsub}[1]{\ensuremath{n_{#1}}\xspace}  % to add a subscript in n
\newcommand{\ptjone}{\ensuremath{\ptsup{\jone}}\xspace}
\newcommand{\ptjtwo}{\ensuremath{\ptsup{\jtwo}}\xspace}
\newcommand{\etajone}{\ensuremath{\etasub{\jone}}\xspace}
\newcommand{\etajtwo}{\ensuremath{\etasub{\jtwo}}\xspace}
\newcommand{\njet}{\ensuremath{\nsub{\mathrm{jet}}}\xspace}
\newcommand{\nbjet}{\ensuremath{\nsub{\Pqb}}\xspace}
\newcommand{\nBoostedTop}{\ensuremath{\nsub{\Pqt}}\xspace}
\newcommand{\nBoostedV}{\ensuremath{\nsub{\PVec}}\xspace}

% Angular variables
\newcommand{\omegaHat}{\ensuremath{\hat{\omega}_{\text{min}}}\xspace}
\newcommand{\omegaTilde}{\ensuremath{\tilde{\omega}_{\text{min}}}\xspace}
\newcommand{\minChi}{\ensuremath{\chi_{\text{min}}}\xspace}
\newcommand{\mindphiAB}[2]{\ensuremath{\Delta\phi_{\text{min}}(#1, \ #2)}\xspace}
\newcommand{\mindphiJetMet}{\ensuremath{\mindphiAB{\text{j}}{\MET}}\xspace}
\newcommand{\dphiTj}{\ensuremath{\mindphiAB{\text{j}_{1, 2}}{\MET}}\xspace}
\newcommand{\dphiFj}{\ensuremath{\mindphiAB{\text{j}_{1, 2, 3, 4}}{\MET}}\xspace}

% Semi-visible jets variables. Taken from AN-19-061
%\newcommand{\metmt}{\ensuremath{\MET/\mt}\xspace}
\newcommand{\metmt}{\ensuremath{R_{\mathrm{T}}}\xspace}
\renewcommand{\PZprime}{\ensuremath{{\PZ}^{\prime}}\xspace}
\newcommand{\PZprimesup}[1]{\ensuremath{{\PZ}^{\prime#1}}\xspace}
\newcommand{\mZprime}{\ensuremath{m_{\PZprime}}\xspace}
\newcommand{\sigmaZprime}{\ensuremath{\sigma_{\PZprime}}\xspace}
\newcommand{\mDark}{\ensuremath{m_{\mathrm{dark}}}\xspace}
\newcommand{\aDark}{\ensuremath{\alpha_{\mathrm{dark}}}\xspace}
\newcommand{\lamDark}{\ensuremath{\Lambda_{\mathrm{dark}}}\xspace}
\newcommand{\aDarkPeak}{\ensuremath{\aDark^{\text{peak}}}\xspace}
\newcommand{\lamDarkPeak}{\ensuremath{\lamDark^{\text{peak}}}\xspace}
\newcommand{\aDarkHigh}{\ensuremath{\aDark^{\text{high}}}\xspace}
\newcommand{\aDarkLow}{\ensuremath{\aDark^{\text{low}}}\xspace}
\newcommand{\rinv}{\ensuremath{r_{\mathrm{inv}}}\xspace}
\newcommand{\Pqdark}{\ensuremath{\chi}\xspace}
\newcommand{\mqdark}{\ensuremath{m_{\Pqdark}}\xspace}
\newcommand{\Paqdark}{\ensuremath{\overline{\chi}}\xspace}
\newcommand{\PqdarkO}{\ensuremath{\chi_{1}}\xspace}
\newcommand{\PqdarkT}{\ensuremath{\chi_{2}}\xspace}
\newcommand{\Pgdark}{\ensuremath{\cPg_{\text{dark}}}\xspace}
\newcommand{\Ppidark}{\ensuremath{\pi_{\text{dark}}}\xspace}
\newcommand{\PpidarkDM}{\ensuremath{\Ppidark^{\text{DM}}}\xspace}
\newcommand{\Prhodark}{\ensuremath{\rho_{\text{dark}}}\xspace}
\newcommand{\PrhodarkDM}{\ensuremath{\Prhodark^{\text{DM}}}\xspace}
\newcommand{\gq}{\ensuremath{g_{\cPq}^{\PZprime}}\xspace}
\newcommand{\gqdark}{\ensuremath{g_{\Pqdark}^{\PZprime}}\xspace}
\newcommand{\Nc}{\ensuremath{N_{c}}\xspace}
\newcommand{\Nf}{\ensuremath{N_{f}}\xspace}
\newcommand{\mb}{\ensuremath{m_{\cPqb}}\xspace}
\newcommand{\mc}{\ensuremath{m_{\cPqc}}\xspace}
\newcommand{\mq}{\ensuremath{m_{\cPq}}\xspace}
\newcommand{\Nstable}{\ensuremath{N_{\text{stable}}}\xspace}
\newcommand{\Nunstable}{\ensuremath{N_{\text{unstable}}}\xspace}
\newcommand{\dijetDeta}{\ensuremath{\Delta\eta(j_{1},j_{2})}\xspace}
\newcommand{\dijetMindphi}{\ensuremath{\Delta\phi_{\mathrm{min}}(j_{1,2}, \met)}\xspace}
\newcommand{\schannel}{\ensuremath{s\text{-channel}}\xspace}
\newcommand{\tchannel}{\ensuremath{t\text{-channel}}\xspace}

% Background processes
\newcommand{\lostlepton}{\ensuremath{\ell_{\text{lost}}}\xspace}
\newcommand{\ztonunu}{\ensuremath{\PZ \rightarrow \Pgn\Pagn}\xspace}
\newcommand{\ztomumu}{\ensuremath{\PZ \rightarrow \Pgm\Pgm}\xspace}
\newcommand{\ztonunupjets}{\ensuremath{\PZ(\rightarrow \Pgn\Pagn) \plusjets}\xspace}
\newcommand{\wtolnupjets}{\ensuremath{\PW(\rightarrow \ell\Pgn) \plusjets}\xspace}
\newcommand{\ttbarpjets}{\ensuremath{\ttbar \plusjets}\xspace}

% Software
\newcommand{\madgraph}{\MADGRAPH}
\newcommand{\MADGRAPHFULL}{\MGvATNLO\,2.6.0\xspace}
\newcommand{\FEYNRULES}{\textsc{FeynRules}\xspace}
\newcommand{\madanalysis}{\textsc{MadAnalysis}\xspace}
\newcommand{\rivet}{\textsc{Rivet}\xspace}

% Other (physics-related)
\newcommand{\transfac}{\ensuremath{\mathcal{T}}\xspace}
\newcommand{\TF}{\transfac}
\newcommand{\BR}{\ensuremath{\mathcal{B}}\xspace}
\newcommand{\BRof}[1]{\ensuremath{\BR(#1)}\xspace} % branching ratio of a process
\newcommand{\OrderOf}[1]{\ensuremath{\mathcal{O}(#1)}\xspace} % (on the) order of something
\newcommand{\eV}{\text{e\kern-0.15ex V}\xspace}
\newcommand{\LSP}{\PSneutralinoOne}
\newcommand{\comruntwo}{\ensuremath{\sqrt{s} = 13\TeV}\xspace}
\newcommand{\higgstoinv}{\ensuremath{\PH \rightarrow \text{inv.}}\xspace}
